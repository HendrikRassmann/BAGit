\chapter{Einleitung}
\label{chap:ein}


%--WW2 Operations Research?\\
%--scheduling\\
Die effiziente Zuteilung von Rechenaufgaben auf unterschiedliche Ressourcen ist nicht erst seit dem Zeitalter der Cloud wichtig. Dies war schon ein kritischer Bereich von Operations Research
% (WW2 OpsResearch Paper einfügen, idealerweise mit turing und so bla bla)
, als ''Computer'' noch ein Beruf war.
In der Ära der Mainframe Computer wurden simple Wartelisten, die von OperatorInnen betreut wurden, nach und nach durch ausgeklügelte Systeme ersetzt. %(einfügen: ich hätte gerne so eine "history of Computing" mäßige Quelle).
Heute übernimmt diese Aufgabe auf fast jedem Computer ein Betriebssystem. Dieses stellt sicher, dass alle Prozesse mit Rechenzeit versorgt werden. Gleichzeitig wird dem Benutzer eine niedrige Antwortzeit seiner Eingaben geboten wird. Sobald mehrere Rechner in einem Cluster zusammen an Aufgaben arbeiten, verschieben sich die Prioritäten. Ein Rechencluster kann nicht einfach wie ein Computer mit vielen Prozessorkernen behandelt werden. Zum einen ist eine vorzeitige Unterbrechung einer Berechnung entweder sehr teuer oder technisch kompliziert, fordert die Planung und Programmierung von Checkpoints \cite{IPS15}, oder ist gar unmöglich \cite{adams1979hitchhiker}.
Zum anderen muss die Antwortzeit des Systems bei neuen Eingaben nicht im Millisekundenbereich liegen.
Außerdem kann sich die Rechenleistung der einzelnen Knoten eines Clusters um einen beliebig großen Faktor unterscheiden.
Des Weiteren können Aufträge strikte Beschränkungen haben, welche und  wie viele Knoten sie benötigen, um ausgeführt werden zu können. Aus Zeit und (Oportunitäts) Kosten sind Experimente an echten Rechenclustern höchstens zur Validierung von Forschungsergebnissen durchführbar sind. Aus diesen Gründen werden solche Systeme in der Regel simuliert.\\
In dieser Arbeit werden zunächst relevante Grundlagen geschildert \ref{chap:methodik}.
Danach wird ein solches Simulationsmodell nachgebaut und erweitert. Als Basis dafür dient \emph{A comparative study of online scheduling algorithms for networks of workstations} von Olaf Arndt et al. \cite{Arn99}. In jenem Paper wird ein Modell vorgestellt, simuliert, ausgewertet und experimentell validiert. Das vorgestellte Modell soll hier nachgebildet werden \ref{reproduktion}. Zur Überprüfung der Nachbildung werden Simulationsläufe analysiert, und die Ergebnisse mit den Ergebnissen in \cite{Arn99} verglichen \ref{vergleich}.\\
Darüber hinaus wird das Modell erweitert, indem Kombinationen von Scheduling Algorithmen durch die Sichtweise der \emph{Funktionalen Programmierung} betrachtet werden \ref{chap:higher-order}.
Zusätzlich sollen verschieden Algorithmen zur Zuteilung von Rechenknoten und Aufträgen\footnote{Im Folgenden werden diese auch als Scheduling Algorithmen oder Scheduler(s) bezeichnet.} statistisch verglichen werden \ref{spt-higher-order}. Durch Modellchecking Verfahren werden diese intuitiv anschaubar gemacht \ref{proptest}. Dadurch ergibt sich eine neue Möglichkeit, Scheduler zu vergleichen, ohne die Performance in einer konkreten Situation zu messen. Anschließend wird überprüft, wie gut die so gewonnenen Intuitionen das tatsächliche Verhalten vorhersagen \ref{prop-sim}.\\

%Da der angehängte Programmcode die üblichen Englischen Bezeichnungen verwendet, werden in dieser Arbeit auf Deutsch vhingegenorgestellte Konzepte zusätzlich zu üblicher Mathematischer Notation auch immer einmal auf Englisch betitelt.


