\chapter{Methodik}
\label{chap:ein}

\section{Problemstellung}

\subsection{Auftrag (Job)}
Ein Auftrag (im Englischen "job") ist ein Prozess, der Arbeitszeit einer oder mehrerer Maschinen benötigt, um abgeschlossen zu werden. Im Folgenden werden verschieden Charakterisierungen eines Auftrags $p$ vorgestellt, nämlich seine

\begin{description}
	\item[Von Schedule unabhängige Eigenschaften] \hfil \\
\textbf{Dauer} (processing time) $p_j$: Dauer die der Auftrag auf Maschinen mit genormter Arbeitsgeschwindigkeit benötigt um abgeschlossen zu werden.
\textbf{Einreihung} (queueing time) $q_j$:Der Zeitpunkt, zudem ein Auftrag bekannt wird.
\textbf{Parallelität} (degree of parallelism) $\pi_j$: Die Anzahl an zugewiesenen Maschinen, die ein Auftrag benötigt, um gestartet zu werden.
\item[Durch Schedule bestimmte Eigenschaften]\hfill \\ 
\textbf{adf}(completion time) $c_j$: dafadsfadsf
\item[weitere, hier nicht beachtet]
\end{description}
Darüberhinaus,auch etc, hier aber nicht relevant


\subsection{MachineEnv}
badbakl

Problemstellung, einleitung in Kurz

Notation aus

\subsection{Zielfunktionen}

\paragraph{Makespan}

Weiterhin stellt er fest, dass dieser von den Charakteristiken der betrachteten Bildersammlung abhängig ist. 
\paragraph{avgWaiting Time}
Da im anvisierten Anwendungsszenario kein annotierter Trainingsdatensatz vorliegt, um die Deskriptorgröße zu bestimmen, 
stellt er in \cite[Kap. 5.4.1]{LDiss} eine Methode zur unüberwachten Deskriptorgrößenschätzung vor.

\subsection{Online, Offline}

\subsection{Scheduling Algorithmen}

\paragraph{FIFO}
Schriftzeichen setzen sich aus zwei Elementen zusammen: Keilen und Winkelhaken. 

\paragraph{SPT}
In der vorliegenden Arbeit soll ein Wordspotting"=System betrachtet werden, welches von \text{Rothacker}
in \cite{LDiss} vorgestellt wurde. Es arbeitet auf der Basis von Bag"=of"=Features"=Hidden"=Markov"=Modellen, kurz
BoF-HMMs.



\section{Simulation}

dasfsaefsafaefdsa

\subsection{Discrete Event Simulation}

dsafeas

\paragraph{biste diskret mann}
dsafdsafasb

\subsection{statistisch auswerten}

voll krass viel

\subsection{Modelchecking}

bla bla bla quickcheck und soll

