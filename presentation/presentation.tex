% % -*- coding:utf-8 -*-
\documentclass[aspectratio=169,10pt]{beamer}
\nonstopmode

\usepackage{appendixnumberbeamer}
\usepackage{graphicx}
\usepackage{url}
\usepackage{pdfpages}
\usepackage{mathtools}
\usepackage{amsmath}
\usepackage{wasysym}
\input{colors}

% \usepackage{beamerthememetropolis}
\usetheme[progressbar=frametitle,noslidenumbers]{metropolis}
\newcommand{\themename}{\textbf{\textsc{metropolis}}\xspace}


\usepackage{xcolor}



\title{Pr\"asentation Bachelor Arbeit:\\
 Simulation von Online Scheduling-Algorithmen f\"ur parallele Systeme}
% \subtitle{Learning to Identify Similarities between Mathematical Expressions}
\author{Hendrik Rassmann}

\institute{TU Dortmund University - Department of Computer Science}
\titlegraphic{\hfill\includegraphics[height=8mm]{tu-do-logo.pdf}}

\date{November 02, 2020}

\begin{document}

\setbeamercovered{transparent}

\maketitle
\Large

\begin{frame}{\"Ubersicht}
\tableofcontents
\end{frame}


\begin{frame}[fragile]{Startpunkt}

\begin{columns}
	\begin{column}{0.57\paperwidth}
		\vspace{0.5pt}
		\includegraphics[width=\linewidth, clip]{images/Arndt}
	\end{column}
	\begin{column}[c]{0.43\paperwidth}
		\begin{itemize}
			\item \alert{Network}
			\item \alert{Online}
			\item \alert{Experiments} ... \alert{confirm}
		\end{itemize}
	\end{column}
\end{columns}
\end{frame}
\section{Problemstellung}


\begin{frame}[t,fragile]{Problemstellung}
Ausgehend von:

\begin{itemize}[<+->]
	\item Rechner/\emph{Nodes} unterschiedlicher \emph{Geschwindigkeiten} bilden zusammen ein (Rechen)\emph{Cluster} 
	\item Auftr\"age/\emph{Jobs} kommen im laufenden Betrieb rein (\emph{online})
	\item Auftr\"age unterscheiden sich bez. \emph{Bearbeitungszeit}, \emph{Anzahl} an ben\"otigten Knoten und \emph{Ankunftszeitpunkt}
\end{itemize}
\end{frame}


\begin{frame}[t,fragile]{Zielfunktionen \footnote{Kriterien von Arndt. et al.}}
\begin{itemize}[<+->]
	\item \emph{makespan}
	\item \emph{average flow time}
	\item \emph{average waiting time} 	
	\item \emph{maximum waiting time}
\end{itemize}
\end{frame}

\begin{frame}[t, fragile]{Scheduling-Algorithmen (Referentiell Transparent)\footnote{Auswahl von Arndt. et al.}} 
$Scheduling$-$Algorithmus : Warteliste \rightarrow Auftrag$\\
\pause

\begin{itemize}[<+->]
	\item First in First out (\emph{FiFo})
	\item Shortest Processing Time first (\emph{SPT})
	\item Greates Processing Time first (\emph{GPT})
\end{itemize}
\end{frame}



%\includegraphics[page=1, scale=0.4]{SchedulingBeispiel.pdf}
\setbeamercolor{background canvas}{bg=}
\includepdf[pages=2-]{SchedulingBeispiel.pdf}

\begin{frame}[t, fragile]{Scheduling-Algorithmen (Referentiell Intransparent)}
\begin{itemize}[<+->]
	\item Random
	\item \alert{First Fit} (W\"ahle den am l\"angsten wartende Auftrag, der sofort gestartet werden kann)
	\item \alert{Backfilling} (W\"ahle einen startbaren Auftrag, der wenn er jetzt gestartet wird, terminiert bevor der am l\"angsten wartende Auftrag starten wird)
	\item \alert{Sichtweise der Funktionalen Programmierung}: $FirstFit(Q) = FiFo ( Filter (Q))$
\end{itemize}
\end{frame}

\begin{frame}[t, fragile]{Beispiel Backfilling}
\end{frame}
\setbeamercolor{background canvas}{bg=}
\includepdf[pages=-]{SchedulingBeispiel2.pdf}


%\section{Simulationsaufbau}
%\begin{frame}[t, fragile]{Simulations Paradigma\footnote{Matloff, Norm: Introduction to discrete-event simulati-
%		on and the simpy language. In: Davis, CA. Dept of Com-
%		puter Science. University of California at Davis. Retrieved on
%		August 2 (2008), Nr. 2009, S. 1–33}}
%	\begin{itemize}
%		\item Aktivit\"atsorientiert (Wettersystem, Physik Simulation in Videospielen)
%		\item \alert{Ereignisorientiert} (Fahrplan)
%		\item Prozessorientiert (Parallele Programmierung)
%	\end{itemize}
%\end{frame}

\begin{frame}[t, fragile]{Simulation und Auswertung}
	\begin{itemize}
		\item Arndt et al. \alert{designen 10 Experimente}, um unterschiedliches Verhalten der Algorithmen zu demonstrieren
		\item Variieren einen Parameter und vergleichen gew\"ahlte Zielfunktionen
		\item ''Bester Algorithmus'' wird \alert{qualitativ} bestimmt. Make span am wichtigsten
	\end{itemize}
\end{frame}

\begin{frame}[fragile]{Beispiel}

\begin{columns}
	\begin{column}{0.57\paperwidth}
		\vspace{0.5pt}
		\includegraphics[width=\linewidth, clip]{images/Figure_2_2}
	\end{column}
	\begin{column}[c]{0.43\paperwidth}
		\begin{itemize}
			\item 250 Auftr\"age
			\item 10 Knoten
			\item timespan - Sp\"atester Ankunftszeitpunkt wird variiert
			\item Mittel von 100 L\"aufen
		\end{itemize}
	\end{column}
\end{columns}
\end{frame}

\section{Eigene Untersuchungen}

\begin{frame}[t,fragile]{Optimistic Backfilling}
	W\"ahle den am l\"angsten wartenden Auftrag, falls startbar. Sonst w\"ahle einen Auftrag, der terminiert, bevor FiFo startet, ODER einen Auftrag, der weniger Knoten ben\"otigt, als \"ubrig bleiben werden, wenn FiFo startet.
\end{frame}

\begin{frame}[t,fragile]{Optimistic Backfilling immer besser?}
\small
\begin{verbatim}	
	queueintT, processingT, realProcessingT, degreeOfParallelism
	id: 0, qT: 1, pT: 3, rPT: 3, doP: 1
	id: 1, qT: 0, pT: 3, rPT: 3, doP: 2
	id: 2, qT: 0, pT: 1, rPT: 1, doP: 2
	id: 3, qT: 0, pT: 1, rPT: 1, doP: 3

	maximumLateness of fifo\_optimistic: 4
	[0]:112|-3
	[1]:112|-3
	[2]:-00|03
	
	maximumLateness of fifo\_backfill: 3
	[0]:112|3--|-
	[1]:11-|300|0
	[2]:--2|3--|-
\end{verbatim}
\end{frame}


\begin{frame}[t, fragile]{Konservativ vs Optimistisch}
	\begin{itemize}[<+->]
		\item 
		''Results are that the performance (...) are practically identical.''	\footnote{Feitelson, Dror G. ; Weil, Ahuva M.: Utilization and pre-
			dictability in scheduling the IBM SP2 with backfilling. In:
			Proceedings of the First Merged International Parallel Proces-
			sing Symposium and Symposium on Parallel and Distributed
			Processing IEEE, 1998, S. 542–546}
		\item''This implies that there is significant difference between normalized distributions of job runtime estimates, and actual
		running times.''\footnote{Keleher, Peter J. ; Zotkin, Dmitry ; Perkovic, Dejan: At-
			tacking the bottlenecks of backfilling schedulers. In: Clus-
			ter Computing 3 (2000), Nr. 4, S. 245–254}
	\end{itemize}
\end{frame}

\begin{frame}[t,fragile]{Gegenbeispiele Automatisch Finden}
	\begin{itemize}[<+->]
		\item \alert{Anschauliche} Beispiele bauen Intuition auf
		\item Am besten ''For Free''
		\item L\"osung: \alert{Property Based Testing}
	\end{itemize}
\end{frame}

\begin{frame}[t, fragile]{Property Based Testing \footnote{Hughes, John ; Claessen, Koen: Quickcheck: a light-
		weight tool for random testing of haskell programs. In:
		Proceedings of the Fifth ACM SIGPLAN International Confe-
		rence on Functional Programming, ICFP, 2000, S. 268–279}}
\begin{itemize}[<+->]
	\item Funktion: \emph{reverse}
	\item Eigenschaft: $reverse(x) == x$
	\item Zuf\"allige Eingaben: [] , [1], [2,2], [4,2]\lightning
	\item Verkleiern (l\"oschen): [4,2] $\rightarrow$ [4] \lightning
	\item Element Verkleinern: [4,2] $\rightarrow$ [1,2] $\rightarrow$ ... $\rightarrow$ [0,1]
\end{itemize}
\end{frame}

\begin{frame}[t, fragile]{PBT f\"ur Simulation}
\begin{itemize}[<+->]
	\item Testen der Implementation: FiFo nie besser als Konservatives Backfilling
	\item Monotonit\"at: System trifft schlechtere  Entscheidungen, mit mehr Informationen?
	\item Schneller Aufbau von Intuition bez. neuer Algorithmen
\end{itemize}
\end{frame}

\begin{frame}[t, fragile]{Neue Algorithmen}
\begin{itemize}[<+->]
	\item ''-Fit'': Smallest-Fit, Greatest-Fit
	\item (Optimistisches)-Backfilling: Smallest-Backfill, Greatest-Backfill
	\item (Optimistisches)-Backfilling nach zwei Funktionen; W\"ahle besten Kandidaten nach Funktion 1, falls nicht startbar, w\"ahle nach Funktion 2, ohne besten Kandidaten zu benachteiligen
\end{itemize}
\end{frame}
\begin{frame}[t, fragile]{Vergleich mit Simulation}
	\begin{columns}
		\begin{column}{0.5\paperwidth}
			\vspace{0.5pt}
			\includegraphics[width=\linewidth, clip]{images/Figure_7_1}
		\end{column}
		\begin{column}[c]{0.5\paperwidth}
			\includegraphics[width=\linewidth, clip]{images/Figure_7_2}
		\end{column}
	\end{columns}
\end{frame}

\begin{frame}[t,fragile]{Qualitative Analyse nicht mehr M\"oglich}
	\begin{itemize}[<+->]
		\item 80 Scheduling Algorithmen kombinierbar
		\item Zu jedem Ergebnis l\"asst sich ein passendes Experiment finden
		\item Algorithmen ohne konkretes Experiment vergleichen?
	\end{itemize}
\end{frame}

\begin{frame}[fragile]{PBT-Score}
		\vspace{0.5pt}
		\includegraphics[width=\linewidth, clip]{images/SchedulerX.png}
\end{frame}
\begin{frame}[fragile]{PBT-Score}
\vspace{0.5pt}
\includegraphics[width=\linewidth, clip]{images/SchedulerXScore.png}
\end{frame}

\begin{frame}[t, fragile]{Vergleich PBT-score und Simulation\footnote{Rangkorrelations nach Spearman: -0.92 wenn mit Experiment 6 von Artndt et al. verglichen}}
\begin{columns}
	\begin{column}{0.5\paperwidth}
		\vspace{0.5pt}
		\includegraphics[width=\linewidth, clip]{images/Figure_5_1}
	\end{column}
	\begin{column}[c]{0.5\paperwidth}
		\begin{table}[]
			\resizebox{\textwidth}{!}{\begin{tabular}{lllll}
					\hline
					algorithm                         & area under curve & pbt score         & a.u.c. Rank & pbt Rank \\ \hline
					fifo      						  & 25862.6322330097 & 0.756218905472637 & 1           & 9        \\
					spt                               & 23798.8361165049 & 0.666666666666667 & 2          & 10        \\
					lpt                               & 23054.2840776699 & 0.765625          & 4           & 8        \\
					Fifo-fit                          & 21039.4811650485 & 1.20338983050847  & 8           & 3        \\
					Fifo-back 						  & 21378.4883018868 & 0.866554054054054 & 5           & 8        \\
					Fifo-optim                        & 21008.5106796117 & 1.28070175438597  & 9           & 2        \\
					Spt-fit                           & 21211.5052427184 & 1.03174603174603  & 6           & 6        \\
					Spt-back                          & 23685.2388349515 & 1.08196721311475  & 7           & 5        \\
					Lpt-fit                           & 21073.0885436893 & 1.12727272727273  & 3           & 4        \\
					Lpt-back                          & 20213.0893203883 & 1.48333333333333  & 10          & 1       \\ \hline
			\end{tabular}}
		\end{table}
	\end{column}
\end{columns}
\end{frame}

\section{Konklusion}
\begin{frame}[t,fragile]{Konklusion}
	\begin{itemize}[<+->]
		\item Funktionale Sichtweise n\"utzlich
		\item PBT n\"utzlich, (fast) umsonst
		\item Zu jedem Ergebnis ist ein Experiment generierbar
		\item Algorithmen in beliebigen Situationen vergleichen problematisch
		\item Experimente aus der Realit\"at ableiten
		\item Algorithmen k\"onnen ohnke konkreten Fall verglichen werden
	\end{itemize}
\end{frame}

%-------------------------------------------
\end{document}
