% Add Packages
% -------------------------------------------------------------------
% URLs
\usepackage{url}

% -------------------------------------------------------------------
% Misc Utility
\usepackage[cmex10]{amsmath}
\usepackage[ruled,linesnumbered]{algorithm2e}
%\usepackage{algorithmicx}
\usepackage{array}
\usepackage{mdwmath}
\usepackage{mdwtab}
\usepackage{eqparbox}

% -------------------------------------------------------------------
% Grafikpakete einbinden
\usepackage{amssymb}
\usepackage{flafter}
\usepackage{listings}
\usepackage{subfig}


% -------------------------------------------------------------------
% Korrekte Darstellung der Umlaute
\usepackage[T1]{fontenc}
\usepackage{ae,aecompl}
\usepackage[utf8]{inputenc}
\usepackage[english,german,ngerman]{babel}
\usepackage{microtype}
	
% -------------------------------------------------------------------
% Anführungszeichen
\usepackage[babel,german=quotes]{csquotes}

%Grafiken ungefähr im selben bereich
\usepackage{placeins}

% Classic thesis:
\usepackage{classicthesis-ldpkg} 
% Options for classicthesis.sty:
\usepackage[eulerchapternumbers,subfig,eulermath,pdfspacing,parts,dottedtoc]{classicthesis}

% -------------------------------------------------------------------
% Setup and Finetuning
\newlength{\abcd} % for ab..z string length calculation
\newcommand{\myfloatalign}{\centering} % how all the floats will be aligned
\setlength{\extrarowheight}{3pt} % increase table row height

% -------------------------------------------------------------------
%Equation and figure numbering
\numberwithin{equation}{section} 
%sets equation numbers <chapter>.<section>.<index>
\numberwithin{table}{section} 
%sets equation numbers <chapter>.<section>.<index>
\numberwithin{figure}{section} 
%sets figure numbers <chapter>.<section>.<index>

% -------------------------------------------------------------------
% Captions look and feel
\captionsetup{format=hang,font=small}


% -------------------------------------------------------------------
% TikZ
\usepackage{tikz}
\usetikzlibrary{automata, positioning, arrows}
\usepackage{pgfplots}
\usepackage{graphicx}
\usepackage{caption}
\usepackage{subfig}
%\usepackage{subcaption}

\pgfplotsset{%
	plotstyle-unten/.style={%
		grid=major, 
		grid style={dotted,gray!25}, % Set the style
		xlabel={Deskriptorgröße $g$}, 
		ylabel={mAP [\%]},
		%every node near coord/.style={font=\footnotesize},
		legend style=
		{at={(0.5,-0.2)},anchor=north,legend columns=-1,column sep=1ex,font=\small} % Put the legend below the plot
	}
}


\pgfplotsset{%
	plotstyle-neben/.style={%
		grid=major, % Display a grid
		grid style={dashed,gray!30}, % Set the style
		%width=0.7\textwidth,
		%height=0.525\textwidth,
		%label style={font=\small},
		%tick label style={font=\small},
		%scaled x ticks=false,
		%scaled y ticks=false,
		xlabel={Deskriptorgröße $g$}, 
		ylabel={mAP [\%]},
		%legend style={at={(0.5,-0.2)},anchor=north,legend columns=-1,column sep=1ex,font=\small}, % Put the legend below the plot
		legend style={at={(1.05,0.5)},anchor=west,font=\small},
		legend cell align=left,
		%x tick label style={rotate=90,anchor=east} % Display labels sideways
		%every node near coord/.style={font=\footnotesize},
		/pgf/number format/.cd,
		use comma,
		1000 sep={\,}
	}
}

\pgfplotsset{%
	plotstyle2/.style={%
		grid=major, % Display a grid
		grid style={dashed,gray!30}, % Set the style
		width=0.7\textwidth,
		height=0.525\textwidth,
		label style={font=\small},
		tick label style={font=\small},
		scaled x ticks=false,
		scaled y ticks=false,
		%legend style={at={(0.5,-0.2)},anchor=north,legend columns=-1,column sep=1ex,font=\small}, % Put the legend below the plot
		legend style={at={(1.05,0.5)},anchor=west,font=\small},
		legend cell align=left,
		%x tick label style={rotate=90,anchor=east} % Display labels sideways
		every node near coord/.style={font=\footnotesize},
		/pgf/number format/.cd,
		use comma,
		1000 sep={\,}
	}
}


% -------------------------------------------------------------------
% Optional, auf jeden Fall evaluieren! \TODO
\usepackage{flafter} 
%\usepackage[]{algorithm2e}